
\chapter{Detector Operations}
By the end of this quarter, Long Shutdown 1 of the LHC had come to an end.  CMS completed its shutdown activities, closed up the detector, and took cosmic ray data to commission the experiment.   In March this included running with the detector solenoid at 3.8 Tesla.  At this time the experiment is ready to continue commissioning with beams to prepare for physics running.

\subsection{Milestones and Metrics}
US CMS has developed a set of milestones and metrics for 2015 to measure performance.   At the present time the detector is still being commissioned and so we do not report metrics.   Milestone progress is reported for each subsystem individually below.


\subsection{BRIL}
The main emphasis of the US-CMS effort as part of the BRIL sub-detector group is the pixel 
luminosity telescope (PLT). The detector is now completely installed inside the CMS detector 
and its functionality has been tested. It is ready to take ``splash'' events that are planned 
during the first/second week of April. Monitoring and online error diagnostics tools are prepared and deployed to enable
regular shift operations. The relative luminosity measurements during VdM scans are in preparation 
and on track for the first scans. 

\begin{table}[htp]
\caption{BRIL Milestones}
\begin{center}
\begin{tabular}{|l|l|r|r|}
\hline
Subsystem&Description&Scheduled&Achieved\\
\hline
BRIL & Hardware installed& Jan & Jan\\
\hline
BRIL& Ready to deliver Lum& March & March \\
\hline
BRIL & Ready to deliver bkg nums& May & \\
\hline
\end{tabular}
\end{center}
\label{BRILMIlestones}
\end{table}


\subsection{Tracker}
Since closing of CMS the tracker (Strips and Pixel) have operated well at the lower temperatures (see
Milestone table). Unlike the conditions during Run 1, the lower humidity conditions have been maintained
even during full magnetic field. This is a good success for the tracker humidity campaign. Detailed, post insertion,
calibrations were performed for both the Barrel (BPiX) and Forward (FPiX) Pixel detectors, and the whole CMS
tracker was able to profit from cosmic ray running with magnetic field on and off. With the field
on, it was found that a sector, about 3\%, of the BPiX will trip if all the modules in that sector are turned on.
Debugging of the BPiX sector, and optimizing the number of powered modules in sector will continue when
the CMS magnetic field is turned back on.
\begin{table}[htp]
\caption{Tracker Milestones}
\begin{center}
\begin{tabular}{|l|l|r|r|}
\hline
Subsystem&Description&Scheduled&Achieved\\

Tracker & Installation and checkout& & Achieved\\
\hline
Tracker & Tracker operate -15C & & Achieved\\
\hline
Tracker & Pixel operate -10C & & Achieved\\
\hline
Tracker& Ready for proton beams& March & March\\
\hline
\end{tabular}
\end{center}
\label{TrackerMilestones}
\end{table}


\subsection{ECAL}
A two-day detailed review of ECAL online and offline readiness for Run 2 was held on 2nd/3rd February 2015. This covered the current commissioning status,  the near-term goals and the plans for commissioning and calibration with the first LHC beams. Significant progress was observed in all areas and the detector is in good shape for Run 2. All parts of ECAL (EB/EE/ES) are participating in global runs, following the migration to the new central DAQ and TCDS systems. The immediate goals involve the recommissioning of the electron/photon trigger path and the validation of the ECAL links to the legacy and upgrade calorimeter trigger systems (with new components installed during LS1). The timing synchronization of the ECAL trigger and readout is being validated and will be further tested during the beam commissioning period (including the use of beam splash events). Laser calibration data is being recorded at 3.8T to monitor the recovery of crystal transparency during LS 1. A successful test of the cold operation of the ECAL preshower, which will operate at $-8^{0}$C during Run 2, has been performed.
\begin{table}[htp]
\caption{ECAL Milestones}
\begin{center}
\begin{tabular}{|l|l|r|r|}
\hline
Subsystem&Description&Scheduled&Achieved\\
\hline
ECAL & Finish HV Install& Feb & delayed\\
\hline
ECAL & Baseline levels zero suppression& March & March \\
\hline
ECAL & Complete install HV calib system & April &\\
\hline
ECAL & Selective readout& June & \\
\hline
ECAL & Trigger thresholds & June & \\
\hline
ECAL & Zero suppression thresholds & June & \\
\hline
\end{tabular}
\end{center}
\label{ECALMilestones}
\end{table}

Regarding the delayed milestone, 3 of the 6 CAEN HV mainframes have been installed. Final one is scheduled to be installed on May 12th. This is behind the original schedule but it is
not a problem because the existing mainframes are fully operational so we are swapping them out progressively. The delay was caused by when we 
received them from CAEN.


\subsection{HCAL}
Since the previous report, the HCAL has completed its Long Shutdown 1 (LS1)  activities and is currently finalizing its Run 2 preparations, which include the completion of the first milestone of the Phase-1 upgrade of installing the HF  $\mu$TCA back-end electronics. Significant effort was invested into the development of the HCAL local reconstruction code for LHC operation with 25ns bunch spacing, including proper out-of-time pile up subtraction. The code was provided for Offline while its faster version for use in the High-Level Trigger (HLT) is presently undergoing tests.
\begin{table}[htp]
\caption{HCAL Milestones}
\begin{center}
\begin{tabular}{|l|l|r|r|}
\hline
Subsystem&Description&Scheduled&Achieved\\
\hline
HCAL& Fully functional HCAL in CRAFT runs & March & March\\
\hline
HCAL& prepared to do HF Phase scan & &\\
&and $\phi$ symmetry calibration analysis& May& \\
\hline
HCAL&New HBHE backend operating in& &\\
& parallel with legacy system& July &\\
\hline
\end{tabular}
\end{center}
\label{HCALMilestones}
\end{table}


\subsection{EMU}
The CSC system participated in cosmic global runs throughout
this period, including extended runs in the closed
configuration at zero magnetic field (CRUZET) and runs with
the solenoid ramped up to full field of 3.8 T (CRAFT). The
operation of the CSC were generally smooth, with a few
remaining firmware/software issues being worked on.
The CSCValidation program was revived to present prompt
diagnostic information from the CRUZET and CRAFT runs. The web
interface was updated and features were added to compare data
from different runs in real time.
The spatial resolution for each chamber type was evaluated from the
2015 CRAFT cosmic ray data. It was found to be essentially unchanged
with respect to the November 2014 data, when the magnet was last powered.
\begin{table}[htp]
\caption{EMU Milestones}
\begin{center}
\begin{tabular}{|l|l|r|r|}
\hline
Subsystem&Description&Scheduled&Achieved\\
EMU& CSC ready for collisions& May & \\
\hline
EMU& Calibration for HLT and & &\\
& Offline included in DB & July & \\
\hline
EMU & Fine timing adjustments & & \\
&with collision data completed & July & \\
\hline
\end{tabular}
\end{center}
\label{EMUMilestones}
\end{table}


\subsection{DAQ}
The DAQ2 system, including new Storage Manager Disk system and the new Trigger Control and Distribution System (TCDS) that was read out through the SLinkExpress fiber link of the FEROL module system was successfully used in global data taking through out the first quarter of FY 2015 in its basic functionality. The focus during this quarter was on commissioning the DAQ2 system to handle edge cases, improving it's the monitoring and performance.
\begin{table}[h]
\caption{DAQ Milestones}
\begin{center}
\begin{tabular}{|l|l|r|r|}
\hline
Subsystem&Description&Scheduled&Achieved\\
\hline
DAQ& Hardware Installation of DAQ2& & \\
& with new HLT nodes complete& April & \\
\hline
DAQ& Complete DAQ2 is operational & & \\
&for collisions& July & \\
\hline
DAQ&$\mu$TCA DAQ link commissioned & & \\
&for new trigger and HCAL FEDs& July & \\
\hline
DAQ&DAQ2 with Run I design performance & September & \\
\hline
\end{tabular}
\end{center}
\label{DAQMilestones}
\end{table}


\subsection{Trigger}
During this quarter the US groups continued their work on the regional calorimeter trigger (RCT) and the endcap muon trigger.  In both significant progress was made timing in the various elements of the triggers.   
This work will continue with the arrival of beams which, being synchronous, simplify the task.  During the cosmic ray running the system was successfully used to trigger the events.
\begin{table}[h]
\caption{Trigger Milestones}
\begin{center}
\begin{tabular}{|l|l|r|r|}
\hline
Subsystem&Description&Scheduled&Achieved\\
\hline
TRIG&Legacy RCT ready for physics&June &\\
\hline
TRIG&MPC ready for physics&June &\\
\hline
TRIG&CSCTF Ready for physics&June & \\
\hline
TRIG& Stage-1 Layer-1 calorimeter trigger&&\\
& ready for physics &September&\\
\hline
\end{tabular}
\end{center}
\label{TriggerMilestones}
\end{table}
